\documentclass[11pt, oneside]{article}   	% use "amsart" instead of "article" for AMSLaTeX format
\usepackage{geometry}                		% See geometry.pdf to learn the layout options. There are lots.
\geometry{letterpaper}                   		% ... or a4paper or a5paper or ... 
%\geometry{landscape}                		% Activate for rotated page geometry
%\usepackage[parfill]{parskip}    		% Activate to begin paragraphs with an empty line rather than an indent
\usepackage{graphicx}				% Use pdf, png, jpg, or eps§ with pdflatex; use eps in DVI mode
								% TeX will automatically convert eps --> pdf in pdflatex		
\usepackage{amssymb}
\usepackage[utf8]{inputenc}
\usepackage[bulgarian]{babel}
\usepackage{tikz}
\usetikzlibrary{decorations.pathreplacing}
\usetikzlibrary{patterns}

%SetFonts

%SetFonts

\title{Лексически Анализ чрез Бимашини}
% \author{Денис Кяшиф}
% \date{Feb 23, 2020}							% Activate to display a given date or no date

\begin{document}
% \maketitle

\section{Крайни Автомати}

\textbf{Дефиниция 1.1} \emph{Краен автомат} дефинираме като петорка \( \mathcal{A} = \langle \Sigma, Q, I, F, \Delta \rangle \), където

\begin{itemize}
	\item \( \Sigma \) е \emph{крайна азбука от символи}
	\item \( Q \) е \emph{крайно множество от състояния}
	\item \( I \subseteq Q \) е \emph{множество от начални състояния}
	\item \( F \subseteq Q \) е \emph{множество от финални състояния}
	\item \( \Delta \subseteq Q \times \Sigma \times Q \) е \emph{релация на преходите}
\end{itemize}
 
Тройки от вида \( \langle q_1, m, q_2 \rangle \in \Delta \) наричаме \emph{преходи} и казваме, че започва състояние \( q_1 \), има етикет \( m \) и завършва в състояние \( q_2 \). Алтернативно, тези преходи обозначаваме като \( q_1 \to^m q_2 \).

\paragraph{Дефиниция 1.2} Нека \( \mathcal{A} \) е краен автомат. \emph{Разширена релация на преходите} \( \Delta^* \subseteq Q \times \Sigma \times Q \) дефинираме индуктивно:

\begin{itemize}
	\item \( \langle q, \epsilon, q \rangle \in \Delta^* \) за всяко \( q \in Q \)
	\item \( \langle q_1, wa, q_2 \rangle \in \Delta^* \) за всяко \( q_1, q_2, q \in Q \), \( a \in \Sigma, w \in \Sigma^* \), ако \( \langle q_1, w, q \rangle \in \Delta^* \) и \( \langle q, a, q_2 \rangle \in \Delta \)
\end{itemize}

\paragraph{Дефиниция 1.3} Нека \( \mathcal{A} = \langle \Sigma, Q, I, F, \Delta \rangle \) е краен автомат. \emph{Път} в \( \mathcal{A} \) наричаме крайна редица от преходи с дължина \( k > 0 \) \[ \pi = q_0 \to^{a_1} q_1 \to^{a_2} \ldots \to^{a_k} q_k \] където \( \langle q_{i-1}, a_i, q_i \rangle \in \Delta \) за \( i = 1 \ldots k \). Казваме, че \emph{пътят} започва от състояние \( q_0 \) и завършва в състояние \( q_k \). Елементите \( q_0,q_1, \ldots ,q_k \) наричаме \emph{състояния на пътя}, а думата \( w = a_1 a_2 \ldots a_k \) наричаме \emph{етикет на пътя}. \newline \emph{Успешен път} в автомата е \emph{път}, който започва от начално състояние и завършва във финално състояние.

\paragraph{Дефиниция 1.4} Нека \( \mathcal{A} \) е краен автомат. Множеството от етикети на всички възможни успещни пътища в \( \mathcal{A} \) наричаме \emph{език на \( \mathcal{A} \)} и обозначаваме като \( L(\mathcal{A}) \). \[ L(\mathcal{A}) = \{ w \in \Sigma^* \hspace{0.1cm} | \hspace{0.1cm} \exists i \in I, f \in F : \langle i, w, f \rangle \in \Delta^* \} \]

\paragraph{Дефиниция 1.5} Нека \( \mathcal{A}_1 \) и \( \mathcal{A}_2 \) са крайни автомати. Казваме, че \( \mathcal{A}_1 \) е еквивалентен на \( \mathcal{A}_2 \) (\( \mathcal{A}_1 \equiv \mathcal{A}_2 \)), ако езиците им съвпадат (\( L(\mathcal{A}_1) = L(\mathcal{A}_2) \))

\section{Детерминистични Крайни Автомати}

Kраен автомат \( \mathcal{A} = \langle \Sigma, Q, s, F, \delta \rangle \) наричаме \emph{детерминистичен}, ако 

\section{Бимашини}

\section{Токенизиращи Релации}

\subsection{Контексти и застъпване}

\begin{tikzpicture}
	\draw (0,0) rectangle (3,1); % u_1
	\draw (0,0) rectangle (5,1); % u_2
	\draw[pattern=horizontal lines] (3,0) rectangle (7,1); % v_1
	\draw[pattern=vertical lines] (5,0) rectangle (8,1); % v_2
	\draw (7,0) rectangle (11,1); % w_1
	\draw (8,0) rectangle (11,1); % w_2
	\draw [decorate,decoration={brace,amplitude=10pt,raise=3pt},yshift=0pt]
		(0,1) -- (3,1) node [black,midway,xshift=0cm,yshift=0.75cm] {\footnotesize $u_1$};
	\draw [decorate,decoration={brace,amplitude=10pt,mirror,raise=3pt},yshift=0pt] 
		(0,0) -- (5,0) node [black,midway,xshift=0cm,yshift=-0.75cm] {\footnotesize $u_2$};
	\draw [decorate,decoration={brace,amplitude=10pt,raise=3pt},yshift=0pt]
		(3,1) -- (7,1) node [black,midway,xshift=0cm,yshift=0.75cm] {\footnotesize $v_1$};
	\draw [decorate,decoration={brace,amplitude=10pt,mirror,raise=3pt},yshift=0pt]
		(5,0) -- (8,0) node [black,midway,xshift=0cm,yshift=-0.75cm] {\footnotesize $v_2$};
	\draw [decorate,decoration={brace,amplitude=10pt,raise=3pt},yshift=0pt]
		(7,1) -- (11,1) node [black,midway,xshift=0cm,yshift=0.75cm] {\footnotesize $w_1$};
	\draw [decorate,decoration={brace,amplitude=10pt,mirror,raise=3pt},yshift=0pt]
		(8,0) -- (11,0) node [black,midway,xshift=0cm,yshift=-0.75cm] {\footnotesize $w_3$};
\end{tikzpicture}

\end{document}