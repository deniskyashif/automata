\documentclass[11pt, oneside]{article}   	% use "amsart" instead of "article" for AMSLaTeX format
\usepackage{geometry}                		% See geometry.pdf to learn the layout options. There are lots.
\geometry{letterpaper}                   		% ... or a4paper or a5paper or ... 
%\geometry{landscape}                		% Activate for rotated page geometry
%\usepackage[parfill]{parskip}    		% Activate to begin paragraphs with an empty line rather than an indent
\usepackage{graphicx}				% Use pdf, png, jpg, or eps§ with pdflatex; use eps in DVI mode
								% TeX will automatically convert eps --> pdf in pdflatex		
\usepackage{amssymb}
\usepackage[utf8]{inputenc}
\usepackage[bulgarian]{babel}
\usepackage{tikz}
\usetikzlibrary{decorations.pathreplacing}
\usetikzlibrary{patterns}

%SetFonts

%SetFonts

\title{Лексически Анализ чрез Бимашини}
% \author{Denis Kyashif}
% \date{Jan 23, 2020}							% Activate to display a given date or no date

\begin{document}
% \maketitle

\section{Крайни Автомати}

Краен автомат дефинираме като петорка \( A = <\Sigma, Q, I, F, \Delta> \), където

\subsection{Детерминирани Крайни Автомати}

Детерминиран краен автомат наричаме \( A = <\Sigma, Q, q_0, F, \delta> \)

\section{Бимашини}

\section{Токенизиращи Релации}

\subsection{Лексическа граматика}

\subsection{Контексти и застъпване}

\begin{tikzpicture}
	\draw (0,0) rectangle (3,1); % u_1
	\draw (0,0) rectangle (5,1); % u_2
	\draw[pattern=horizontal lines] (3,0) rectangle (7,1); % v_1
	\draw[pattern=vertical lines] (5,0) rectangle (8,1); % v_2
	\draw (7,0) rectangle (11,1); % w_1
	\draw (8,0) rectangle (11,1); % w_2
	\draw [decorate,decoration={brace,amplitude=10pt,raise=3pt},yshift=0pt]
		(0,1) -- (3,1) node [black,midway,xshift=0cm,yshift=0.75cm] {\footnotesize $u_1$};
	\draw [decorate,decoration={brace,amplitude=10pt,mirror,raise=3pt},yshift=0pt] 
		(0,0) -- (5,0) node [black,midway,xshift=0cm,yshift=-0.75cm] {\footnotesize $u_2$};
	\draw [decorate,decoration={brace,amplitude=10pt,raise=3pt},yshift=0pt]
		(3,1) -- (7,1) node [black,midway,xshift=0cm,yshift=0.75cm] {\footnotesize $v_1$};
	\draw [decorate,decoration={brace,amplitude=10pt,mirror,raise=3pt},yshift=0pt]
		(5,0) -- (8,0) node [black,midway,xshift=0cm,yshift=-0.75cm] {\footnotesize $v_2$};
	\draw [decorate,decoration={brace,amplitude=10pt,raise=3pt},yshift=0pt]
		(7,1) -- (11,1) node [black,midway,xshift=0cm,yshift=0.75cm] {\footnotesize $w_1$};
	\draw [decorate,decoration={brace,amplitude=10pt,mirror,raise=3pt},yshift=0pt]
		(8,0) -- (11,0) node [black,midway,xshift=0cm,yshift=-0.75cm] {\footnotesize $w_3$};
\end{tikzpicture}

\section{Крайни Преобразуватели}

\end{document}